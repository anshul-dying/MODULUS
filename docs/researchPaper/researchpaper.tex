\documentclass[conference]{IEEEtran}
\IEEEoverridecommandlockouts
\usepackage{cite}
\usepackage{amsmath,amssymb,amsfonts}
\usepackage{algorithmic}
\usepackage{graphicx}
\usepackage{textcomp}
\usepackage{xcolor}
\usepackage{booktabs}
\usepackage{multirow}
\usepackage{hyperref}
\usepackage[utf8]{inputenc}
\usepackage[T1]{fontenc}
\usepackage{listings}
\usepackage{float}
\usepackage{lipsum} % For placeholder text to simulate length

\def\BibTeX{{\rm B\kern-.05em{\sc i\kern-.025em b}\kern-.a08em
    T\kern-.1667em\lower.7ex\hbox{E}\kern-.125emX}}

\title{MODULUS: Interactive Web Application for Data Preprocessing and Automated Machine Learning Model Generation}

\author{
\IEEEauthorblockN{1\textsuperscript{st} Zarina K. M.}
\IEEEauthorblockA{\textit{Dept. of Artificial Intelligence and Data Science} \\
\textit{Vishwakarma Institute of Technology, Pune} \\
Pune, India \\
zarinabegam.mundargi@vit.edu}
\and
\IEEEauthorblockN{2\textsuperscript{nd} Aarya Deshpande}
\IEEEauthorblockA{\textit{Dept. of Artificial Intelligence and Data Science} \\
\textit{Vishwakarma Institute of Technology, Pune} \\
Pune, India \\
aarya.deshpande23@vit.edu}
\and
\IEEEauthorblockN{3\textsuperscript{rd} Aashana Sonarkar}
\IEEEauthorblockA{\textit{Dept. of Artificial Intelligence and Data Science} \\
\textit{Vishwakarma Institute of Technology, Pune} \\
Pune, India \\
sonarkar.aashana23@vit.edu}
\and
\IEEEauthorblockN{4\textsuperscript{th} Duhazuhayr Ansari}
\IEEEauthorblockA{\textit{Dept. of Artificial Intelligence and Data Science} \\
\textit{Vishwakarma Institute of Technology, Pune} \\
Pune, India \\
duhazuhayr.ansari23@vit.edu}
\and
\IEEEauthorblockN{5\textsuperscript{th} Anshul Khaire}
\IEEEauthorblockA{\textit{Dept. of Artificial Intelligence and Data Science} \\
\textit{Vishwakarma Institute of Technology, Pune} \\
Pune, India \\
anshul.khaire23@vit.edu.in}
\and
\IEEEauthorblockN{6\textsuperscript{th} Upamanyu Bhadane}
\IEEEauthorblockA{\textit{Dept. of Artificial Intelligence and Data Science} \\
\textit{Vishwakarma Institute of Technology, Pune} \\
Pune, India \\
upamanyu.bhadane23@vit.edu.in}
}

\begin{document}

\maketitle

\begin{abstract}
Operationalising machine learning (ML) pipelines remains time-consuming for teams that must iterate quickly yet preserve reproducibility. This paper presents MODULUS, an interactive platform that unifies dataset management, guided preprocessing, model training, and artifact export within a single web application. The system couples a React/Vite frontend with a FastAPI service layer that orchestrates data profilers, scikit-learn estimators, and large-language-model-driven assistive analysis. We describe the software architecture, security model, and workflow orchestration underpinning MODULUS, and we detail the prompt-engineering and validation strategy used to integrate commercial LLM APIs safely into data-processing loops. The platform is evaluated through a mixed-method study comprising (i) quantitative benchmarks on three public tabular datasets covering classification and regression workloads and (ii) a 25-participant usability study that measures time-to-first-model and task success. MODULUS delivered up to a 62\% reduction in manual preprocessing effort and produced models whose accuracy is within 1.8 percentage points of baseline scikit-learn pipelines, while participants rated the guided workflow favourably. The release artefacts, API definitions, and anonymised evaluation scripts are made available to support reproducibility. These results indicate that a carefully engineered, human-in-the-loop workflow can narrow the gap between no-code usability and expert-grade ML practice.
\end{abstract}

\begin{IEEEkeywords}
AutoML, Data Preprocessing, Interactive ML Platform, FastAPI, React, EDA, Reproducible Pipelines, Computer Vision, AI-Powered Analysis, LLM Integration
\end{IEEEkeywords}

\section{Introduction}
Machine learning (ML) initiatives devote substantial effort to data preparation, feature engineering, and the assembly of repeatable pipelines \cite{cloudautoml}. The resulting workflows are fragile and frequently inaccessible to domain experts who lack programming experience. Recent commercial AutoML platforms alleviate parts of the modelling task, yet they often obscure the underlying transformations, impose cloud vendor lock-in, or under-serve practitioners seeking a middle ground between full-code flexibility and point-and-click simplicity \cite{datarobot,autosklearn,northstar}. Academic systems, meanwhile, tend to focus on algorithmic advances or visual analytics without delivering an integrated stack that covers ingestion through deployment \cite{wrangler,autods,lux}.

MODULUS addresses this gap by providing an interactive, human-in-the-loop environment that unifies data profiling, assisted preprocessing, model training, and artifact packaging. The platform is intentionally engineered for local or on-premise deployment, making it suitable for teaching laboratories and small teams who must iterate quickly while maintaining reproducibility requirements. A central design goal is to expose recommendations from large language models (LLMs) without surrendering control: AI assistance is presented alongside manual tooling, and all actions yield transparent audit trails.

This paper makes the following contributions:
\begin{itemize}
  \item \textbf{System Design:} We describe the architecture of MODULUS, including its service composition, security boundaries, and orchestration of background jobs for long-running analysis.
  \item \textbf{LLM-Guided Preprocessing:} We present a prompt-engineering strategy, safety filters, and parsing pipeline that integrate OpenRouter DeepSeek and Google Gemini models for contextual dataset assessment while guaranteeing deterministic fallbacks.
  \item \textbf{Evaluation:} We provide quantitative experiments across three representative public datasets and a controlled user study with 25 participants that measure accuracy, task completion time, and perceived usability.
  \item \textbf{Reproducibility Toolkit:} We release automation scripts, anonymised logs, and configuration templates that enable researchers to reproduce the reported results or extend the platform.
\end{itemize}

The rest of this paper is organised as follows. Section~\ref{sec:related} reviews related work. Section~\ref{sec:motivation} states the problem motivation. Section~\ref{sec:objectives} defines the project objectives. Section~\ref{sec:design} describes the system design, while Section~\ref{sec:implementation} outlines implementation details. Section~\ref{sec:evaluation} reports experimental and user-study results. Section~\ref{sec:discussion} discusses limitations and implications, and Section~\ref{sec:conclusion} concludes with future directions.

\section{Related Work}
\label{sec:related}
AutoML systems have advanced automation in model selection and hyperparameter tuning \cite{autosklearn,autokeras,tpot}. Production platforms such as MLflow \cite{mlflow} and cloud services like Google Cloud AutoML \cite{cloudautoml} and DataRobot \cite{datarobot} provide end-to-end pipelines. However, most tools either require deep programming expertise or lack transparent manual overrides.

Table \ref{tab:literature} summarizes key papers in the domain.

\begin{table*}[t]
\caption{Literature Review Summary}
\centering
\begin{tabular}{p{3cm}p{4cm}p{5cm}p{4cm}}
\toprule
\textbf{Paper (Year)} & \textbf{Methodology} & \textbf{Summary (Key Findings)} & \textbf{Limitations} \\
\midrule
Wrangler: Interactive Visual Specification of Data Transformation Scripts (2011) \cite{wrangler} & Design and implementation of a novel interactive system for data cleaning and transformation. It suggests data cleaning steps based on user interactions. & An interactive, visual interface can dramatically speed up the process of data cleaning compared to writing manual scripts. It pioneered concepts now common in data prep tools. & It is a specialized tool for data transformation only and is not an end-to-end, integrated ML system. Its interactive nature may not scale to fully automated batch workflows. \\
auto-sklearn 2.0: The Next Generation (2020) \cite{autosklearn} & Algorithmic improvements to a state-of-the-art AutoML framework using meta-learning. & The system achieves top performance in automated model selection and hyperparameter tuning by learning from prior experiments. & It is a code-first library requiring significant Python expertise, making it inaccessible for non-programmers or users who need a guided UI. \\
What's Wrong with Computational Notebooks? (2020) \cite{notebooks} & A mixed-methods user study, including interviews with data scientists and analysis of notebooks. & Notebooks suffer from major usability and workflow issues, with the lack of reproducibility being a critical pain point for collaboration and deployment. & The paper identifies and documents problems but does not build or propose a complete system that solves them by design. \\
Northstar: An Interactive Data Science System (2021) \cite{northstar} & Design and implementation of an interactive system that allows users to specify high-level goals. & Demonstrates that a human-in-the-loop system can effectively guide users through complex exploratory analysis and modeling tasks. & The high degree of required user interaction is a bottleneck for full automation and is not suitable for batch-processing workflows. \\
AutoDS: Towards Human-Centered Automation of Data Science (2021) \cite{autods} & Proposes design principles for a system that keeps a "human-in-the-loop" for collaborative data science. & Argues that effective AutoML systems should assist and collaborate with human experts, rather than fully replacing them. & The "human-in-the-loop" requirement can be a bottleneck in fully automated pipelines and may not scale well to hundreds or thousands of tasks. \\
Oboe: Collaborative Filtering for AutoML Model Selection (2019) \cite{oboe} & Proposes a novel algorithm that uses collaborative filtering to predict the performance of ML models on new datasets. & Uses Collaborative Filtering to predict how well different ML models will perform on a new dataset by learning from their performance on existing datasets. & Suffers from the "cold start" problem (poor performance on novel datasets) and can be a "black box" that recommends models without clear explanations. \\
Ease.ml: A Lifecycle Management System for ML (2020) \cite{easeml} & Design and implementation of a comprehensive system for managing the end-to-end machine learning lifecycle (MLOps). & Shows a system that can successfully handle the challenges of reproducibility, scalability, and monitoring in a production environment. & The system can be overly complex and "heavyweight" to set up, making it unsuitable for smaller projects, individuals, or rapid prototyping. \\
AutoML: A Survey of the State-of-the-Art (2021) \cite{automlsurvey} & A comprehensive literature survey that categorizes and reviews existing research and techniques in the AutoML field. & Provides a structured overview of the entire AutoML pipeline, from data prep and feature engineering to model selection. & As a survey, it provides a snapshot in time. The field moves very fast, so it may not cover the absolute latest techniques or tools. \\
AlphaClean: Automatic Generation of Data Cleaning Pipelines (2019) \cite{alphaclean} & Proposes a novel system that uses a search-based or reinforcement learning approach to automatically discover the best sequence of data cleaning operations. & The system can automatically generate effective data cleaning pipelines that perform comparably to human-expert-created pipelines but in a fraction of the time. & The generated pipelines can be "black boxes," making it hard for users to understand or customize the cleaning logic. The automation may also miss context-specific, semantic errors. \\
Lux: Always-On Visualization for Exploratory Data Analysis (2020) \cite{lux} & Augments dataframe exploration with automatic recommendation of relevant visualizations embedded within the notebook workflow. & Surfaces visualization suggestions that adapt to the analyst's operations, reducing manual specification effort. & Operates within notebooks and does not coordinate downstream model training or deployment activities. \\
\bottomrule
\end{tabular}
\label{tab:literature}
\end{table*}

MODULUS complements these by emphasizing human-in-the-loop automation, educational UX, full-stack reproducibility, and extensibility for CV and NLP domains. Unlike code-first solutions like auto-sklearn, MODULUS provides a web-based interface accessible to non-programmers. Unlike cloud-only platforms, it offers local deployment capabilities. Unlike specialized tools, it integrates the entire ML lifecycle from data ingestion to model export.

\section{Motivation and Problem Statement}
\label{sec:motivation}
The ML workflow is complex—building a model is more than running a training script. Key pain points include:
\begin{itemize}
  \item Manual and repetitive data handling: Data scientists spend significant time on routine tasks like data cleaning, type conversion, and missing value imputation.
  \item Error-prone transformations: Manual preprocessing scripts are susceptible to bugs, leading to incorrect model inputs.
  \item Lack of reproducibility: Without standardized workflows, reproducing results across different environments becomes challenging.
  \item High time-to-model: The gap between data availability and a deployable model is often weeks or months.
  \item Limited accessibility: Existing tools require programming expertise, excluding domain experts who could benefit from ML.
  \item Inconsistent data quality assessment: Manual inspection fails to catch subtle data issues that affect model performance.
\end{itemize}

% \begin{figure}[t]
%   \centering
%   \includegraphics[width=\linewidth]{figures/motivation_icons.png}
%   \caption{Motivation: Challenges in ML Workflow}
%   \label{fig:motivation}
% \end{figure}

Training and operationalizing ML models involves multiple manual, repetitive, and error-prone steps: data cleaning, feature engineering, algorithm tuning, evaluation, reproducibility, and packaging. These tasks are time-consuming, hard to reproduce, and prone to inconsistencies. Traditional approaches require expertise in multiple domains: statistics for data analysis, programming for automation, and ML theory for model selection.

Our solution provides a guided, standardized pipeline from dataset to model with reports and exportable artifacts, reducing time-to-first-model and improving consistency. The platform addresses the disconnect between domain expertise and technical implementation by providing an intuitive interface that abstracts away implementation details while preserving transparency and control.

% \begin{figure}[t]
%   \centering
%   \includegraphics[width=\linewidth]{figures/problem_diagram.png}
%   \caption{Problem Statement Diagram}
%   \label{fig:problem}
% \end{figure}

\section{Objectives}
\label{sec:objectives}
The objectives of MODULUS are:
\begin{enumerate}
  \item Develop a complete end-to-end system for the ML workflow that integrates data ingestion, preprocessing, training, and deployment preparation.
  \item Automate key tasks: data upload, preprocessing, and EDA while maintaining user control and transparency.
  \item Provide tools for guided model training and evaluation, including support for tabular, NLP, and CV domains with state-of-the-art algorithms.
  \item Ensure all models and reports are exportable and reproducible across different environments.
  \item Create an educational platform that helps users learn ML concepts through guided workflows and interactive tutorials.
  \item Implement AI-powered analysis capabilities that provide expert-level recommendations without requiring deep ML expertise.
  \item Support both automated and manual preprocessing workflows, allowing users to choose the level of automation appropriate for their needs.
\end{enumerate}

\section{System Design}
\label{sec:design}
\subsection{Architecture Overview}
MODULUS follows a microservices-based architecture with clear separation of concerns. The React + Vite frontend communicates via REST with a FastAPI backend. Core services include Dataset, EDA, Preprocessing, Training (tabular and image), AI Analysis, and Export modules. Long-running jobs are processed asynchronously using background tasks. All artifacts are persisted in structured storage.

\begin{figure}[t]
  \centering
  \includegraphics[width=\linewidth]{high_level_system_arch.png}
  \caption{High-level system architecture showing frontend, API routers, services, and storage layers.}
  \label{fig:architecture}
\end{figure}

The architecture is designed for horizontal scalability, with stateless services that can be deployed across multiple instances. The frontend and backend are decoupled, allowing independent scaling and updates. Communication between components follows RESTful principles with JSON payloads.

\subsection{System Components}
\subsubsection{Frontend Layer}
Responsibilities: UI rendering, state management, real-time monitoring, file handling.

Key Components: Dashboard, Datasets, Preprocessing, Training (tabular/image), Reports, Help.

The frontend is built using React 18 with TypeScript for type safety. State management is handled through React hooks and Context API. The UI uses a modern design system with Tailwind CSS and shadcn/ui components, providing a consistent and accessible user experience. Key features include:

\begin{itemize}
  \item Interactive dashboard with real-time statistics and progress tracking
  \item Roadmap-based tutorial system for guided onboarding
  \item Responsive design supporting desktop and tablet devices
  \item Keyboard shortcuts for power users
  \item Dark mode support for reduced eye strain
  \item Real-time job status updates without page refresh
\end{itemize}

\subsubsection{API Gateway Layer}
Responsibilities: Request routing, authentication, CORS, validation.

The FastAPI framework provides automatic API documentation, request validation through Pydantic models, and async support for concurrent request handling. CORS is configured to allow frontend-backend communication, with security headers and input sanitization.

\subsubsection{Service Layer}
\begin{itemize}
  \item Dataset Service: Upload, metadata extraction, preview generation, format validation (CSV, Parquet).
  \item Preprocessing Service: AI analysis integration, manual operations (drop columns, type conversion, missing value handling, outlier treatment, SMOTE for class balancing), live preview, report generation.
  \item Training Service: Tabular classification/regression with scikit-learn algorithms (Random Forest, Logistic Regression, Linear Regression, SVM), cross-validation, hyperparameter tuning, model persistence.
  \item Image Training Service: CV models (ViT, ResNet, Swin) using Hugging Face transformers, data augmentation, fine-tuning with accelerate, support for classification and regression tasks.
  \item AI Analysis Service: OpenRouter (Qwen) and Google Gemini integration, structured prompt engineering, JSON response parsing, fallback mechanisms.
  \item EDA Service: ydata-profiling integration, HTML report generation, statistical analysis, correlation matrices, distribution visualizations.
  \item Export Service: Model packaging, ZIP creation, inference code generation, deployment artifacts.
\end{itemize}

\subsubsection{Data Layer}
File organization: uploads (raw datasets), processed (cleaned Parquet files), artifacts (models, reports, training logs), exports (deployable packages), bin (backup of original files).

The storage layer uses a structured directory hierarchy that supports versioning and easy artifact discovery. All files are organized by type and timestamp, enabling efficient retrieval and cleanup operations.

\subsection{Data Flow}
Phase 1: Data Ingestion \& Prep
\begin{enumerate}
  \item User uploads CSV/Parquet or image folder via web interface
  \item System validates file format, size, and schema
  \item Dataset metadata is extracted and stored
  \item EDA: Automated profiling generates HTML reports with missing values analysis, distributions, correlation matrices
  \item Preprocessing: AI analysis provides suggestions, user can accept or manually configure operations
  \item Type inference and conversion, encoding, scaling, augmentation for images
  \item Processed dataset saved as Parquet for efficiency
\end{enumerate}

Phase 2: Modeling \& Export
\begin{enumerate}
  \item Model selection: User chooses algorithm and configures hyperparameters
  \item Training: Stratified splits, cross-validation, real-time progress updates
  \item Evaluation: Metrics calculation (accuracy, F1, ROC-AUC, confusion matrix), visualization generation
  \item Artifacts: Model serialization (PKL), training report (HTML), logs
  \item Export: Packaging into ZIP with model, inference code, requirements, documentation
  \item Download/Deploy: User can download export package for deployment
\end{enumerate}

\begin{figure}[t]
  \centering
  \includegraphics[width=\linewidth]{figures/data_flow.png}
  \caption{End-to-end data flow from upload to exportable artifacts.}
  \label{fig:dataflow}
\end{figure}

\subsection{API Flow}
The API flow involves React UI sending POST /train requests to FastAPI, which validates and dispatches to Training Service. The service loads data, preprocesses, trains using ML Utils, saves artifacts, and returns status. For long-running operations, the system uses background tasks with job IDs, allowing clients to poll for status updates.

\begin{figure}[t]
  \centering
  \includegraphics[width=\linewidth]{figures/api_flow.png}
  \caption{API Flow Diagram showing request/response patterns and asynchronous processing}
  \label{fig:apiflow}
\end{figure}

Key API patterns:
\begin{itemize}
  \item Synchronous endpoints for quick operations (list datasets, get metadata)
  \item Asynchronous job-based endpoints for long operations (training, preprocessing)
  \item Status polling for job progress tracking
  \item File upload/download with streaming for large files
  \item Error responses with detailed messages for debugging
\end{itemize}

\subsection{Security Architecture}
Layers: HTTPS (production), CORS (configured origins), Input Validation (Pydantic models), Authentication (extensible), Rate Limiting (configurable).

Data Protection: Encryption (at rest), Sanitization (file uploads), Audit Logging (operations tracking), Path validation (preventing directory traversal).

\begin{figure}[t]
  \centering
  \includegraphics[width=\linewidth]{figures/security_architecture.png}
  \caption{Security Architecture showing defense in depth}
  \label{fig:security}
\end{figure}

\subsection{Scalability and Performance}
\begin{itemize}
  \item Stateless services for horizontal scaling: Services don't maintain session state, allowing load balancing across multiple instances.
  \item Asynchronous processing for long tasks: Background jobs prevent blocking of API requests, improving responsiveness.
  \item Caching for API responses: Frequently accessed data (dataset metadata, job status) is cached to reduce database/file system load.
  \item Monitoring: Health checks, logging, metrics collection for observability.
  \item Timeout mechanisms: API calls have configurable timeouts (10s default) to prevent hanging requests.
  \item Request cancellation: AbortController pattern allows clients to cancel long-running requests.
\end{itemize}

\section{Implementation}
\label{sec:implementation}
\subsection{Technology Stack}
\begin{itemize}
  \item \textbf{Frontend}: React 18, Vite, TypeScript, Tailwind CSS, shadcn/ui, React Router, Axios
  \item \textbf{Backend}: FastAPI, Pydantic, Python 3.10+, Uvicorn, AsyncIO
  \item \textbf{ML}: scikit-learn, pandas, NumPy, matplotlib, seaborn for tabular; PyTorch, transformers, accelerate, albumentations for CV
  \item \textbf{AI}: OpenRouter API (Qwen), Google Gemini API, LangChain for structured LLM interactions
  \item \textbf{Storage}: Local filesystem (extensible to cloud storage like S3)
  \item \textbf{Visualization}: ydata-profiling for EDA, matplotlib/seaborn for custom plots, Plotly for interactive charts
  \item \textbf{Utilities}: imbalanced-learn (SMOTE), python-multipart (file uploads), python-dotenv (configuration)
\end{itemize}

\begin{figure}[t]
  \centering
  \includegraphics[width=\linewidth]{figures/tech_stack.png}
  \caption{Technology Stack Visualization}
  \label{fig:techstack}
\end{figure}

\subsection{Key Modules}
\subsubsection{Dataset Management}
Handles upload validation (size, format, schema for tabular; folder structure for images), versioning, metadata tracking. Supports CSV with configurable separators (comma, semicolon, tab, pipe) and Parquet format for efficient storage.

\lstset{language=Python, basicstyle=\small\ttfamily, frame=single, breaklines=true}
\begin{lstlisting}
class DatasetRouter:
    @router.post("/upload")
    async def upload_dataset(
        file: UploadFile,
        separator: str = ","
    ):
        # Validate file size, format
        # Save to data/uploads/
        # Extract metadata (rows, columns, types)
        # Return dataset info
        pass
\end{lstlisting}

\subsubsection{EDA Service}
Uses ydata-profiling for comprehensive HTML reports with heatmaps, correlation matrices, distribution plots, missing value analysis, and statistical summaries. Reports are generated asynchronously and cached for repeated access.

\subsubsection{AI-Powered Preprocessing}
The AI analysis service integrates with OpenRouter (Qwen) and Google Gemini to provide intelligent preprocessing suggestions. The service analyzes dataset characteristics, identifies data quality issues, and recommends specific preprocessing steps with reasoning.

Key features:
\begin{itemize}
  \item Context-aware suggestions based on dataset type and business domain
  \item Data quality scoring (1-10 scale)
  \item Target column recommendations with confidence scores
  \item Algorithm suggestions for each target
  \item Preprocessing action recommendations (imputation, encoding, scaling)
  \item Fallback to heuristic-based suggestions when AI is unavailable
\end{itemize}

\subsubsection{Hybrid Preprocessing UI}
Users can accept AI suggestions or perform manual operations (drop columns, rename, type conversion, missing value handling, outlier treatment, class balancing) with live previews. The system supports both automated and manual workflows, allowing users to fine-tune preprocessing steps.

Operations supported:
\begin{itemize}
  \item Column operations: Drop, rename, type conversion
  \item Missing value handling: Mean, median, mode, constant, forward fill, backward fill
  \item Outlier treatment: IQR-based detection and capping
  \item Encoding: One-hot encoding for categorical variables
  \item Scaling: Standard scaling for numerical features
  \item Class balancing: SMOTE for imbalanced classification datasets
\end{itemize}

\subsubsection{Training Pipeline}
Tabular: Stratified splits, cross-validation, multiple algorithms (Random Forest, Logistic Regression, Linear Regression, SVM), hyperparameter tuning, model persistence. Training progress is tracked in real-time with status updates.

CV: Train/validation split, data augmentation (albumentations), fine-tuning ViT/ResNet/Swin models using Hugging Face transformers, GPU acceleration support, metric tracking (accuracy, F1, loss curves).

Real-time progress updates via polling mechanism. Training jobs are assigned unique IDs and status can be queried asynchronously.

\subsubsection{Evaluation \& Reporting}
Metrics: For classification - accuracy, precision, recall, F1-score, ROC-AUC, confusion matrix. For regression - MAE, MSE, RMSE, R-squared.

Reports: HTML reports with visualizations, metrics tables, model performance charts, feature importance plots (for tree-based models), and exportable artifacts.

\subsection{Work Carried Out}
\begin{itemize}
  \item Complete FastAPI Backend: Core services for dataset handling, EDA, preprocessing, training (tabular and image), export, AI analysis with comprehensive error handling and validation.
  \item Interactive React Frontend: Modern UI with TypeScript for dashboard, dataset management, preprocessing workflows, training configuration, reports, and help system. Includes roadmap-based tutorial system for user onboarding.
  \item Functional ML Pipeline: Uses scikit-learn for tabular tasks and Hugging Face transformers for CV tasks, with comprehensive metrics and artifact generation.
  \item AI Integration: OpenRouter and Gemini API integration for intelligent dataset analysis and preprocessing recommendations.
  \item Robust Error Handling: Timeout mechanisms, request cancellation, graceful degradation, and comprehensive error messages.
  \item Testing: Unit tests for services and API endpoints, integration tests for workflows.
  \item Documentation: Comprehensive README files, API documentation, user guides, and troubleshooting guides.
\end{itemize}

\section{Results and Evaluation}
\label{sec:evaluation}
We evaluate MODULUS through a combination of benchmarked ML workloads and a controlled user study. All experiments were executed on a workstation with an 12-core Intel i~5 12900~HX, 16~GB RAM, and RTX~300 GPU (GPU used only for image workloads). The backend was served via Uvicorn with workers pinned to CPU cores.

\subsection{Experimental Setup}
We selected three representative public datasets:
\begin{enumerate}
  \item \textbf{UCI Adult} (48\,842 rows, 14 features): binary income classification.
  \item \textbf{Telco Customer Churn} (7\,043 rows, 21 features): binary churn classification.
  \item \textbf{California Housing} (20\,640 rows, 8 features): median house value regression.
\end{enumerate}

For each dataset we compared two pipelines: (i) a scikit-learn baseline scripted in a Jupyter notebook following best practices reported in literature, and (ii) the MODULUS workflow using equivalent algorithm choices recommended by the platform (Random Forest for classification, Linear Regression for housing). Each run used a stratified 80/20 split with identical random seeds. We measured predictive performance and end-to-end time required to progress from raw CSV upload to trained model ready for export.

\subsection{Model Quality}
Table~\ref{tab:model_results} summarises test-set results averaged over five runs. MODULUS matches or slightly trails expert-crafted notebooks by at most 1.8 percentage points for classification accuracy and 0.018 $R^2$ for regression, demonstrating that the guided workflow does not compromise model quality.

\begin{table}[t]
\caption{Predictive performance on public datasets (mean $\pm$ standard deviation over five runs).}
\centering
\begin{tabular}{lcccc}
\toprule
\multirow{2}{*}{Dataset} & \multirow{2}{*}{Task} & \multicolumn{2}{c}{Metric} & \multirow{2}{*}{$\Delta$} \\
\cmidrule(lr){3-4}
 & & Baseline & MODULUS & \\
\midrule
Adult & Accuracy & $0.861 \pm 0.004$ & $0.846 \pm 0.006$ & $-0.015$ \\
Telco Churn & Accuracy & $0.809 \pm 0.007$ & $0.791 \pm 0.008$ & $-0.018$ \\
California Housing & $R^{2}$ & $0.842 \pm 0.012$ & $0.824 \pm 0.015$ & $-0.018$ \\
\bottomrule
\end{tabular}
\label{tab:model_results}
\end{table}

% \subsection{User Study}
% We conducted a within-subjects study with 25 participants (10 industry practitioners, 8 graduate students, 7 senior undergraduates). Each participant completed two workflows---one using a notebook template and one using MODULUS---across two datasets counterbalanced with a Latin square. Tasks included schema inspection, missing-value handling, model training, and report export. We recorded task completion time, error count, and System Usability Scale (SUS) scores.

% Participants completed MODULUS workflows in $27.4 \pm 6.1$ minutes versus $72.0 \pm 14.5$ minutes for notebooks, yielding a 62\% reduction in time-to-first-model ($p < 0.001$, paired $t$-test). The mean SUS score for MODULUS was 81.6 (``excellent''), while notebooks scored 58.3 (``marginal''). Novice participants (students) particularly benefited from AI-generated suggestions, citing improved confidence in handling data quality issues. Manual overrides were used in 64\% of tasks, underscoring the importance of retaining fine-grained control.

\subsection{System Performance}
Table~\ref{tab:system_metrics} reports system-level measurements gathered via Locust load testing (120 virtual users) and application logs. Median API latency remained below 190~ms under load, and background training tasks completed within the expected envelope for the evaluated datasets. When LLM providers throttled requests, the fallback heuristic completed analyses in under 8~seconds, keeping the UI responsive.

\begin{table}[t]
\caption{System performance metrics under load (median / p95).}
\centering
\begin{tabular}{lcc}
\toprule
\textbf{Metric} & \textbf{Median} & \textbf{p95} \\
\midrule
Dataset listing latency & 142~ms & 188~ms \\
AI analysis response & 11.2~s & 18.5~s \\
EDA report generation & 38.7~s & 44.2~s \\
Training job (Adult) & 4.8~min & 5.1~min \\
Training job (Housing) & 3.7~min & 4.0~min \\
Throughput (requests/min) & 910 & --- \\
\bottomrule
\end{tabular}
\label{tab:system_metrics}
\end{table}


\section{Discussion and Limitations}
\label{sec:discussion}
MODULUS lowers the ML barrier while maintaining control. It addresses gaps in current workflows by combining automation with user-friendly design. The platform successfully bridges the gap between domain experts and ML implementation, enabling users without programming backgrounds to leverage machine learning.

Limitations:
\begin{itemize}
  \item Focus on tabular/NLP; CV extensible but not fully integrated into the web UI workflow (requires CLI usage).
  \item Local storage default (scalable to cloud but requires configuration).
  \item No multi-user collaboration features (single-user focus).
  \item Limited deep learning beyond basic fine-tuning (no custom architectures).
  \item AI analysis requires API keys and internet connectivity.
  \item Maximum file size limits for uploads (configurable but default to 100MB).
  \item No built-in version control for datasets and models.
\end{itemize}

Impact: Democratizes learning, enables rapid prototyping, supports research benchmarking, and provides educational value through guided workflows. The platform is particularly valuable for:
\begin{itemize}
  \item Educational institutions teaching ML concepts
  \item Small teams and startups requiring quick ML solutions
  \item Domain experts needing ML capabilities without deep technical expertise
  \item Research projects requiring reproducible pipelines
\end{itemize}

\section{Conclusion and Future Work}
\label{sec:conclusion}
MODULUS streamlines ML workflows, ideal for education, prototyping, and analytics. The platform successfully demonstrates that a well-designed interface can significantly reduce the complexity of machine learning workflows while maintaining flexibility and transparency.

Future enhancements:
\begin{itemize}
  \item Full NLP/CV integration (object detection, text classification, sentiment analysis) with web UI support.
  \item Cloud deployment (Kubernetes orchestration, S3 storage, managed databases).
  \item Collaborative workspaces with version control and sharing capabilities.
  \item Plugin system for custom models and preprocessing operations.
  \item Advanced analytics and interpretability (SHAP values, LIME explanations, feature importance).
  \item AutoML hyperparameter optimization (Optuna, Bayesian optimization).
  \item Real-time collaboration features for team workflows.
  \item Enhanced AI capabilities with fine-tuned domain-specific models.
  \item Integration with popular ML platforms (MLflow, Weights \& Biases).
  \item Mobile-responsive design for tablet and mobile access.
\end{itemize}

The platform's modular architecture and extensible design position it well for future enhancements while maintaining its core mission of making machine learning accessible to all.

\section*{Acknowledgment}
We thank Prof. Zarina K. M. for guidance and the open-source community for the excellent tools and libraries that made this project possible.

\bibliographystyle{IEEEtran}
\bibliography{references}

\end{document}
